%!TEX root =  ../main.tex

\begin{abstract}
近年来,卷积神经网络(CNNs)因其具有高推断精度和强自适应性而被广泛应用于各种领域,例如:计算机视觉、语音识别等。移动手机当前已经成为了人们日常生活中的随身携带之物,其每天都产生着大量与人们相关的传感数据,因此许多工程项目也尝试着将卷积神经网络应用于移动平台来处理这些数据为人类提供更加智能的服务。然而,由于受到当前移动平台的资源限制(内存、计算能力、电池容量等),基于CNN模型的应用并没有在手机移动平台上成为主流。

目前,手机上基于CNN模型的应用绝大部分都是采用“客户端-服务器”模式。然而,这种模式不仅强依赖于网络性能而且会导致用户隐私泄露,所以于手机移动端直接进行卷积神经网络的高能效前向推断已然成为学术界和工业界所需共同面对的严峻挑战。本文直面这些挑战,采用一系列优化策略开发出一套可以高能效运行在Android平台的卷积神经网络推断时库,并将其运用在一个生活日志型应用中,以探索从系统层进一步降低该类场景应用的运行时功耗。论文的主要工作包括:

\begin{enumerate}
\item 基于OpenCL异构编程框架使用手机移动端GPU进行卷积神经网络的前向推断过程。解析Caffe、Tensorflow等深度学习框架训练出来的卷积神经网络模型权重,以便在手机端重构网络,并将卷积、内积等计算密集型的操作转移到手机GPU上进行。
\item 通过对卷积神经网络全连接层权重的剪枝、重训操作,降低网络模型的存储占用。使用稀疏矩阵向量乘代替密集矩阵的内积操作,进一步降低卷积神经网络的前向推断时间。
\item 为了能够充分利用当前以及未来移动设备SoC所提供的异构计算特征,本文提出了一种简单有效的方法,使得基于CNN模型的应用在运行时可以根据其所处运行环境中CPU、GPU等异构处理器的性能差异,自适应地调整CPU、GPU等所分配的计算任务量。
\item 针对基于CNN的生活日志型应用,本文从系统层分析并建模预测这类应用的运行时负载,探索使用动态电压频率调节技术(DVFS)调整CPU/GPU的电压、频率来进一步降低基于CNN模型应用的运行时功耗之可能性。
\end{enumerate}

\keywords{能效;异构计算;权值压缩;卷积神经网络;移动平台}
\end{abstract}

\begin{enabstract}
Convolutional Neural Networks (CNNs) have become more and more powerful in the computer vision domain, as they achieve the state-of-the-art accuracy. Despite this, it is generally difficult to apply CNNs on mobile platforms. Client-server paradigm is a straightforward way to deploy CNNs on mobile phones, but studies have shown that it suffers serious problems, such as privacy leaks. Recently, researchers focus on using heterogeneous local processors (e.g., GPUs, CPUs) to accelerate the inference of CNNs. Utilizing all local processors available can achieve the highest performance, but it might incur energy-inefficiency. Different from previous works, this paper concerns more about energy-efficiency of
CNN-based mobile applications. We present an adaptive strategy, which is able to compute the energy-efficiency of all local processors, and further to obtain the energy-efficient device processor combination to perform CNN inference in parallel. The strategy is implemented on ODROID platform, where the evaluation results show that our proposed approach provides 3.67$\times$ higher energy-efficiency with only 9.7\% performance degradation on average compared with the greedy strategy which tries to use all local processors available.

\enkeywords{Energy-efficiency;Weights compression; Heterogeneous computing; CNNs; Mobile platform}
\end{enabstract}
