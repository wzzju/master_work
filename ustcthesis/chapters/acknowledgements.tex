\begin{acknowledgements}
时光荏苒,三年的研究生学习生涯已悄然接近尾声。三年前的我追随自己内心的兴趣所在毅然选择从其他专业跨入到计算机科学这个领域,三年后的我依然不后悔当初的选择。这三年来,我完成了从对科研的陌生、迷茫到乐在其中的蜕变。期间,我收获到的不仅仅是知识还有道不尽的友情、师生情,而我需要感谢的人也有很多。

首先,我很庆幸三年的学习生活中有着四位导师对我悉心教导。感谢周学海老师,是他将我带入了计算机领域科学研究的大门。周老师为人和蔼可亲,很少见到他发脾气,但是当我们科研不努力时他也会恨铁不成钢地批评我们。正是因为周老师对学生的严格要求,我学会了主动阅读与自己研究工作相关的论文、认真解决科研中遇到的问题。感谢李曦老师,是他每次在我研究工作遇到瓶颈停滞不前时,及时给予我解决问题的思路。李老师对于科研工作一丝不苟的态度一直是我学习的榜样。虽然李老师平时表现得严厉,但是他其实也很幽默,对待学生更是关爱有加。感谢王超老师在研究工作和论文书写上给我提出的建议。超哥每次对我科研工作的点评都是一针见血,让我及时意识到研究过程中存在的问题。感谢陈香兰老师在科研和学习生活上对我的帮助。在科研或生活上遇到解不开的难题时,陈老师的一番话总能让我恍然大悟,给予我思路和信心去解决困难。

其次,我要感谢程志南师兄、宋家臣师兄、周金红师兄、徐友军师兄以及赵洋洋师姐。程志南师兄是我在研究生学习生涯中遇到的贵人。从开题选择到小论文写作,对于我每次的叨扰,南哥都会很有耐心地和我讨论,给我指明方向。感谢宋家臣师兄深夜还在和我探讨我的开题工作,他有条不紊的做事风格十分值得我学习。感谢周金红师兄在我研究生选题时给予我的一些建设性意见。感谢徐友军师兄带领我学习Linux内核,为我以后的科研工作打下了技术基础。感谢赵洋洋师姐给予我科研问题上的一些解决思路。当然,我还要感谢我的同窗好友张奕玮、罗海钊、徐冲冲、孙凡、鲁云涛、郭玲等人,是他们陪我度过了整个研究生学习生活,是他们在我遇到困难时给予我鼓励,是他们让我在科研道路上走得更远。

最后,我要感谢我的爸爸和妈妈。他们虽然都年过半百,但是对于我的求学之路总是无条件支持。我很愧对我的父母,五十几岁本该是他们安享晚年的年龄,但是因为我还在读书,他们仍在不辞劳苦地工作着。祝愿我的父母身体健康,希望我所学之知识能够报答他们的养育之恩。

\begin{flushright}
王震 \\
于2018年4月
\end{flushright}



\end{acknowledgements}
